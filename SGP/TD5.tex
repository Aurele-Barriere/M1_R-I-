\documentclass[a4paper,french,11pt]{article}
\usepackage[T1]{fontenc}
\usepackage[utf8]{inputenc}

\usepackage[french]{babel}
\usepackage{amsmath}
\usepackage{fullpage}


\usepackage{listings}
\usepackage{color}

\definecolor{red}{rgb}{1,0,0.1}
\definecolor{purple}{rgb}{1,0,0.4}
\definecolor{silver}{rgb}{0.7,0.7,0.7}
\definecolor{green}{rgb}{0,0.6,0}


\lstset{language=C++,
  belowcaptionskip=1\baselineskip,
  breaklines=true,
  xleftmargin=\parindent,
  language=C,
  showstringspaces=false,
  basicstyle=\footnotesize\ttfamily,
  keywordstyle=\bfseries\color{green},
  commentstyle=\itshape\color{silver},
  identifierstyle=\color{black},
  stringstyle=\color{red},
  frame=l
}



\title{TD5 SGP : Programmation des entrée/sorties caractère}

\begin{document}

\maketitle

\paragraph{Question 2}

\begin{lstlisting}
void tty_init(void) {
    initialiser();
    autoriser_it_e();
    autoriser_it_s();
}
\end{lstlisting}

\begin{lstlisting}
semahpore buff_s = 1;
int i, lg;
char* buffer_s;


void tty_em_ligne(char* l) {
    P(buff_s);
    autoriser_it_s();
    lg = longueur(l); // avec le '0'
    buffer_s = copy(l);
    i = 1;
    ecrire(l[0]);
}

it_s {
    if (i < lg) {
        ecrire(buffer_s[i])
        i++;
    }
    else {
        interdire_it_s();
        V(buffer_s);
    }
}
\end{lstlisting}

\begin{lstlisting}
semaphore read = 0;
int i = 0;


void tty_rec_ligne(int *lg, char* l) {
    P(read);
    l = copy(buffer);
    autoriser_it_e();
}

it_e {
    lire(e);
    if (c != '\0') {
        buffer[i] = c;
        c++;
    }
    else {
        buffer[i] = '\0';
        interdire_it_e();
        V(read);
        i = 0;
    }
}
\end{lstlisting}


\end{document}
